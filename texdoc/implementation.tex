\documentclass[document.tex]{subfiles}
\begin{document}
\newpage

\section{Реализация}
\subsection{Исходный код}
\begin{flushleft}
  В листинге~1 представлен исходный код описания проекта, включающий в себя
  описание проекта, зависимости и подключаемые пространства имен приложения.
  \lstinputlisting[caption=project.clj]{project.clj}
\end{flushleft}

\begin{flushleft}
  В листинге~2 описывается функция round-robin-map, распределяющая список
  значений full-list поочередно по списку нод distribute-list.
  \lstinputlisting[caption=src/smapreduce/utils.clj]{src/smapreduce/utils.clj}
\end{flushleft}

\begin{flushleft}
  Листинг~3 описывает исходный код пространства имен smapreduce.core, который
  позволяет запустить приложение с аргументами map или reduce.
  \lstinputlisting[caption=src/smapreduce/core.clj]{src/smapreduce/core.clj}
\end{flushleft}

\begin{flushleft}
  В листинге~4 показан исходный код пространства имен smapreduce.server,
  позволяющий открыть сервер nREPL (встроенный интерпретатор Clojure, доступный
  с использованием сетевого соединения).
  \lstinputlisting[caption=src/smapreduce/server.clj]{src/smapreduce/server.clj}
\end{flushleft}

\begin{flushleft}
  Листинг~5 описывает пространство имен smapreduce.client, реализующее функцию
  map-client для соединения с сервером nREPL и функцию get-full-list,
  конкатенирующую списки, возвращаемые с каждой из нод кластера функцией
  map-client в результирующий список.
  \lstinputlisting[caption=src/smapreduce/client.clj]{src/smapreduce/client.clj}
\end{flushleft}

\newpage
\subsection{Запуск программы}
\lstset{language=Bash}

\begin{flushleft}
  Программное обеспечение запускается следующим образом. \\
  Для запуска двух нод на портах 1088 и 1099 нужно выполнить следующие
  команды из корня проекта (до этого должен быть установлен Clojure и
  пакетный менеджер Leiningen):
  \begin{lstlisting}
    # Map REPL instance on port 1088
    $ lein run map 1088
    # Map REPL instance on port 1099
    $ lein run map 1099
  \end{lstlisting}

  Для сложения числа <<3>> с каждым значением списка и сверткой списка через
  оператор сложения с использованием нод на портах 1088 и 1099 необходимо
  выполнить следующую команду:
  \begin{lstlisting}
    # MapReduce client
    $ lein run reduce "+" "#(+ 3 %)" "(1 2 3 4)" 1088 1099
  \end{lstlisting}

\end{flushleft}

\end{document}
