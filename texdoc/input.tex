\documentclass[document.tex]{subfiles}
\begin{document}
\newpage
\pagenumbering{arabic}

\section{Постановка задачи}
\subsection{Описание алгоритма}
\begin{flushleft}
  MapReduce -- это программная модель и ассоциированная с ней реализация
  для обработки и генерации больших объемов данных с использованием параллельного
  распределенного кластерного алгоритма. \\
  Простейшая реализация алгоритма состоит из операции map, которая является
  функцией высшего порядка и применяет заданную функцию к каждому элементу
  входного списка и операции reduce, которая является функцией высшего порядка
  и производит свертку результирующего списка, приводя его к единственному
  атомарному значению. \\
  Операция map может выполняться для любой части списка независимо и может
  быть распределена между нодами кластера. Операция reduce выполняется после
  получения всех результатов выполнения операций map на всех нодах кластера.
\end{flushleft}

\subsection{Описание задачи}
\begin{flushleft}
  Необходимо реализовать имплементацию алгоритма MapReduce на JVM-стеке. \\
  Для реализации выбран язык программирования Clojure, включающий в себя все
  достоинства языка программирования LISP по обработке списков и построенный на
  стеке технологий виртуальной машины Java.
\end{flushleft}


\end{document}
